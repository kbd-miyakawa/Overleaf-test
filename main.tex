\documentclass[
uplatex,%       uplatexを使う
Overleaf=true,% ipaex
dvipdfmx,%      driver
b5ver_for_tokiyotosho/14Q,% 基本体裁
useotf,%   otfパッケージを使う(macros内で読む)
]{kbdbook9}
\usepackage{graphicx}
\usepackage{xcolor}
\usepackage{macros}%%% 設定マクロ類
\usepackage[OT1,T1]{fontenc} %% T1 encoding
\usepackage{textcomp}
\usepackage{lmodern}%% lmodern font
\usepackage{amsmath,amssymb} % ams
\usepackage{float}%% float でHを使う
\usepackage{bm}
\usepackage{bxtexlogo}%% さまざまなTeX関連のLOGOの用意
    \bxtexlogoimport{*,**}
\newcommand{\cs}[1]{{\fontencoding{T1}\texttt{\symbol{92}#1}}}

\usepackage{url}

\usepackage{makeidx}%%%% 坂東追加 2018/7/1
\makeindex%%%% 坂東追加 2018/7/1



\begin{document}
\mainmatter
\LaTeX\index{latex@\LaTeX}のテスト\index{てすと@テスト}%
文献\cite{__2012}を参照.%% 参考文献引用

%% duck!
\begin{figure}
\centering
\OvlfDuck[speech=\textbf{Check!}]
\scalebox{0.5}{\OvlfDuck[think={\textsf{?}}]}%0.5倍 
\caption[]{アヒルの表示.\\
オプションは\url{http://ctan.math.illinois.edu/graphics/pgf/contrib/tikzducks/tikzducks-doc.pdf}参照}
\end{figure}
%% snowman!
\scsnowman[hat=true,muffler=red,arms=true,scale=10]



\backmatter
%% 参考文献
\bibliographystyle{plain}%% 一般的な文献用スタイルファイル
\bibliography{Zotero}    %% .bibの読み込み

%% 索引
\printindex

\end{document}
